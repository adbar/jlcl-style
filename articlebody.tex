\documentclass{jlcl}
%%==========================================================================================%%
%%==========================================================================================%%
%% Contributions by Adrien Barbaresi, Kais Haddar, Osama Hamed.


\title{\articletitle}
\author{AUTHOR NAME\\
AUTHOR INSTITUTION\\
\texttt{AUTHOR EMAIL} \and
AUTHOR 2 NAME\\
AUTHOR 2 INSTITUTION\\
 \texttt{AUTHOR 2 EMAIL}}
 
%%==========================================================================================%%
\begin{document}
\selectlanguage{english}

% arabtex
\setarab % choose the language specific conventions
\vocalize % switch diacritics for short vowels on
\transtrue % additionally switch on the transliteration
% \arabtrue % default


\setcounter{page}{1}
\thispagestyle{firstpage}

\authordata


%%==========================================================================================%%
%%           ARTICLE                                                                        %%
%%==========================================================================================%%

\section*{Abstract}
This article describes the Arabic alphabet.



\section{Introduction}

\subsection{The Alphabet}

%The Arabic alphabet has 29 letters including the Hamza (\textRL{ء}), consonants and three long vowels (\textAR{ا، و، ي}).
%The short vowels or vowel signs are not part of alphabet but they are merely oral.
%Along with the three short vowels (\textRL{ـَ ـُ ـِ}), the Arabic script has other phonetic symbols, together known as diacritic marks.
%Diacritics are sound symbols represented as strokes that are placed above or below the letter.

% % % solution found on http://tex.stackexchange.com/questions/141832/babel-arabic-changes-enumeration-level
This is a test: \RL{b.hr lw.t}

\begin{RLtext}'imtala'ati al-_dAkiraTu al-'iliktrUniyyaTu bi-mo`.tayAtiN `a^swA'iyyaTiN\end{RLtext}

\nocite{*}
\bibliographystyle{apa}
{\small \bibliography{references}}


\end{document}